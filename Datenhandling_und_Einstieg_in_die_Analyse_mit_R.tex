\documentclass[10pt,ngerman,onside]{article}
\usepackage{lmodern}
\usepackage{amssymb,amsmath}
\usepackage{ifxetex,ifluatex}
\usepackage{fixltx2e} % provides \textsubscript
\ifnum 0\ifxetex 1\fi\ifluatex 1\fi=0 % if pdftex
  \usepackage[T1]{fontenc}
  \usepackage[utf8]{inputenc}
\else % if luatex or xelatex
  \ifxetex
    \usepackage{mathspec}
  \else
    \usepackage{fontspec}
  \fi
  \defaultfontfeatures{Ligatures=TeX,Scale=MatchLowercase}
\fi
% use upquote if available, for straight quotes in verbatim environments
\IfFileExists{upquote.sty}{\usepackage{upquote}}{}
% use microtype if available
\IfFileExists{microtype.sty}{%
\usepackage{microtype}
\UseMicrotypeSet[protrusion]{basicmath} % disable protrusion for tt fonts
}{}
\usepackage[margin=1in]{geometry}
\usepackage{hyperref}
\hypersetup{unicode=true,
            pdftitle={Methoden quantitativer Forschung},
            pdfauthor={Dipl.-Math. Norman Markgraf},
            pdfborder={0 0 0},
            breaklinks=true}
\urlstyle{same}  % don't use monospace font for urls
\ifnum 0\ifxetex 1\fi\ifluatex 1\fi=0 % if pdftex
  \usepackage[shorthands=off,main=ngerman]{babel}
\else
  \usepackage{polyglossia}
  \setmainlanguage[]{german}
\fi
\usepackage{color}
\usepackage{fancyvrb}
\newcommand{\VerbBar}{|}
\newcommand{\VERB}{\Verb[commandchars=\\\{\}]}
\DefineVerbatimEnvironment{Highlighting}{Verbatim}{commandchars=\\\{\}}
% Add ',fontsize=\small' for more characters per line
\usepackage{framed}
\definecolor{shadecolor}{RGB}{248,248,248}
\newenvironment{Shaded}{\begin{snugshade}}{\end{snugshade}}
\newcommand{\KeywordTok}[1]{\textcolor[rgb]{0.13,0.29,0.53}{\textbf{#1}}}
\newcommand{\DataTypeTok}[1]{\textcolor[rgb]{0.13,0.29,0.53}{#1}}
\newcommand{\DecValTok}[1]{\textcolor[rgb]{0.00,0.00,0.81}{#1}}
\newcommand{\BaseNTok}[1]{\textcolor[rgb]{0.00,0.00,0.81}{#1}}
\newcommand{\FloatTok}[1]{\textcolor[rgb]{0.00,0.00,0.81}{#1}}
\newcommand{\ConstantTok}[1]{\textcolor[rgb]{0.00,0.00,0.00}{#1}}
\newcommand{\CharTok}[1]{\textcolor[rgb]{0.31,0.60,0.02}{#1}}
\newcommand{\SpecialCharTok}[1]{\textcolor[rgb]{0.00,0.00,0.00}{#1}}
\newcommand{\StringTok}[1]{\textcolor[rgb]{0.31,0.60,0.02}{#1}}
\newcommand{\VerbatimStringTok}[1]{\textcolor[rgb]{0.31,0.60,0.02}{#1}}
\newcommand{\SpecialStringTok}[1]{\textcolor[rgb]{0.31,0.60,0.02}{#1}}
\newcommand{\ImportTok}[1]{#1}
\newcommand{\CommentTok}[1]{\textcolor[rgb]{0.56,0.35,0.01}{\textit{#1}}}
\newcommand{\DocumentationTok}[1]{\textcolor[rgb]{0.56,0.35,0.01}{\textbf{\textit{#1}}}}
\newcommand{\AnnotationTok}[1]{\textcolor[rgb]{0.56,0.35,0.01}{\textbf{\textit{#1}}}}
\newcommand{\CommentVarTok}[1]{\textcolor[rgb]{0.56,0.35,0.01}{\textbf{\textit{#1}}}}
\newcommand{\OtherTok}[1]{\textcolor[rgb]{0.56,0.35,0.01}{#1}}
\newcommand{\FunctionTok}[1]{\textcolor[rgb]{0.00,0.00,0.00}{#1}}
\newcommand{\VariableTok}[1]{\textcolor[rgb]{0.00,0.00,0.00}{#1}}
\newcommand{\ControlFlowTok}[1]{\textcolor[rgb]{0.13,0.29,0.53}{\textbf{#1}}}
\newcommand{\OperatorTok}[1]{\textcolor[rgb]{0.81,0.36,0.00}{\textbf{#1}}}
\newcommand{\BuiltInTok}[1]{#1}
\newcommand{\ExtensionTok}[1]{#1}
\newcommand{\PreprocessorTok}[1]{\textcolor[rgb]{0.56,0.35,0.01}{\textit{#1}}}
\newcommand{\AttributeTok}[1]{\textcolor[rgb]{0.77,0.63,0.00}{#1}}
\newcommand{\RegionMarkerTok}[1]{#1}
\newcommand{\InformationTok}[1]{\textcolor[rgb]{0.56,0.35,0.01}{\textbf{\textit{#1}}}}
\newcommand{\WarningTok}[1]{\textcolor[rgb]{0.56,0.35,0.01}{\textbf{\textit{#1}}}}
\newcommand{\AlertTok}[1]{\textcolor[rgb]{0.94,0.16,0.16}{#1}}
\newcommand{\ErrorTok}[1]{\textcolor[rgb]{0.64,0.00,0.00}{\textbf{#1}}}
\newcommand{\NormalTok}[1]{#1}
\usepackage{graphicx,grffile}
\makeatletter
\def\maxwidth{\ifdim\Gin@nat@width>\linewidth\linewidth\else\Gin@nat@width\fi}
\def\maxheight{\ifdim\Gin@nat@height>\textheight\textheight\else\Gin@nat@height\fi}
\makeatother
% Scale images if necessary, so that they will not overflow the page
% margins by default, and it is still possible to overwrite the defaults
% using explicit options in \includegraphics[width, height, ...]{}
\setkeys{Gin}{width=\maxwidth,height=\maxheight,keepaspectratio}
\IfFileExists{parskip.sty}{%
\usepackage{parskip}
}{% else
\setlength{\parindent}{0pt}
\setlength{\parskip}{6pt plus 2pt minus 1pt}
}
\setlength{\emergencystretch}{3em}  % prevent overfull lines
\providecommand{\tightlist}{%
  \setlength{\itemsep}{0pt}\setlength{\parskip}{0pt}}
\setcounter{secnumdepth}{0}
% Redefines (sub)paragraphs to behave more like sections
\ifx\paragraph\undefined\else
\let\oldparagraph\paragraph
\renewcommand{\paragraph}[1]{\oldparagraph{#1}\mbox{}}
\fi
\ifx\subparagraph\undefined\else
\let\oldsubparagraph\subparagraph
\renewcommand{\subparagraph}[1]{\oldsubparagraph{#1}\mbox{}}
\fi

%%% Use protect on footnotes to avoid problems with footnotes in titles
\let\rmarkdownfootnote\footnote%
\def\footnote{\protect\rmarkdownfootnote}

%%% Change title format to be more compact
\usepackage{titling}

% Create subtitle command for use in maketitle
\newcommand{\subtitle}[1]{
  \posttitle{
    \begin{center}\large#1\end{center}
    }
}

\setlength{\droptitle}{-2em}
  \title{Methoden quantitativer Forschung}
  \pretitle{\vspace{\droptitle}\centering\huge}
  \posttitle{\par}
\subtitle{Datenhandling und Einstieg in die Analyse mit R}
  \author{Dipl.-Math. Norman Markgraf}
  \preauthor{\centering\large\emph}
  \postauthor{\par}
  \predate{\centering\large\emph}
  \postdate{\par}
  \date{Regionale FOM Dozententagen in Frankfurt 2017}

% ===========================================================================
% before.tex
% ---------------------------------------------------------------------------
\usepackage{fancyhdr}
\usepackage{amsmath}
\usepackage{setspace}
\usepackage{courier}
%
\pagestyle{fancy}
%
\rhead{\includegraphics[width = .2\textwidth]{ifes.jpg}}
% ===========================================================================

\begin{document}
\maketitle

% ===========================================================================
% before.tex
% ---------------------------------------------------------------------------
\setstretch{1.44}
\begin{center}
\includegraphics{ifes.jpg}
\end{center}

\tableofcontents
\newpage
% ===========================================================================

\hypertarget{einleitung}{%
\section{Einleitung}\label{einleitung}}

\hypertarget{das-programm-r}{%
\subsection{Das Programm R}\label{das-programm-r}}

\textbf{Warum R?}

\begin{itemize}
\tightlist
\item
  R ist ein Programm für Statistik und Datenanalyse.
\item
  R ist für Linux, MacOS X und Windows (95 oder höher) Plattformen
  verfügbar.
\item
  R ist eine elegante und umfassende statistische und grafische
  Programmiersprache.
\item
  R kann eine steile Lernkurve L haben
  \texttt{(L\ =\ Zeiteinheit/Erfolgseinheit)}.
\item
  R ist kostenlos! Wenn Sie Lehrender oder Studierender sind, sind die
  Vorteile offensichtlich.
\item
  R bietet eine unvergleichliche Plattform für die Programmierung neuer
  statistischer Methoden in einer einfachen und unkomplizierten Weise.
\item
  R enthält fortgeschrittene statistische Routinen, die noch nicht in
  anderen Software-Paketen verfügbar sind.
\item
  R verfügt über state-of-the-art Grafiken Fähigkeiten.
\end{itemize}

\textbf{Warum RStudio?}

RStudio ist eine integrierte Entwicklungsumgebung (IDE), die die
Verwendung von R für Anänfäger und Experten erleichtert.

\textbf{Warum KEIN RCommander?}

Der RCommander ist eine grafische Benutzeroberfläche, allgemein bekannt
als \emph{Rcmdr}, und kann als Paket in \textbf{R} und \textbf{RStudio}
geladen werden. -- Warum sollte man aber?

\textbf{Wo bekomme ich \textbf{R} und \textbf{R Studio}?}

\begin{itemize}
\tightlist
\item
  Die Software R kann von einer der
  \href{https://cran.r-project.org/mirrors.html}{CRAN-Spiegelservern}
  unter \url{https://cran.r-project.org/mirrors.html} heruntergeladen
  werden.
\item
  RStudio bekommen Sie auf der Homepage von
  \href{https://www.rstudio.com}{RStudio} unter
  \url{https://www.rstudio.com}.
\end{itemize}

\textbf{Arbeiten mit \textbf{R} und \textbf{R Studio}}

\begin{itemize}
\item
  Mit RStudio benutzen Sie entweder die Console oder ein Skiptfenster
  zur Eingabe der Skriptbefehle. Ein neues R-Skriptfenster öffen Sie
  über \textbf{File--\textgreater{}New File--\textgreater{}R Script}.
\item
  Wenn Sie ein RStudio-Skipt benutzen, dann gibt es vier Fenster:

  \begin{itemize}
  \tightlist
  \item
    links oben die das Skript,
  \item
    links unten die Console,
  \item
    rechts oben Daten, Objekte und Historie und
  \item
    links unten Dateien, Abbildung, Hilfeseiten und Tipps
  \end{itemize}
\item
  Einzelne Befehle aus dem Skriptfenster in R Studio können Sie annstatt
  mit dem \textbf{Run Button} auch mit \textbf{Strg} und \textbf{Enter}
  an die Console schicken.
\item
  Befehlen können auch direkt in der Console ausgeführt werden
  (Ausführung mi der \textbf{Enter}-Taste).
\item
  Das Arbeiten mit Skript hat einige Vorteile und macht die Datenanalyse
  sehr kompfortabel.
\end{itemize}

\textbf{Funktionalitäten in R}

\begin{itemize}
\tightlist
\item
  Zusätzliche Funktionalitäten können über Zusatzpakete hinzugeladen
  werden. Diese müssen ggf. zunächst installiert werden.
\item
  Paketen könne über den Menüpunkt \textbf{Tools} installiert werden
  oder durch den Befehl \textbf{install.packages(``Names des Pakets'')}.
\item
  Pakete werden geladen mit dem Befehl \textbf{library(Name des
  Pakets)}.
\item
  Achten Sie auch die \textbf{Anführungszeichen} nur beim Installieren.
\item
  Mit der Pfeiltaste nach oben können Sie (nur in der Console) einen
  vorherigen Befehl wieder aufrufen.
\item
  Im R-Skrip können Sie sich die Befehle und Erläuterungen individuell
  einrichten und diese unabhängig von der Reihenfolge immer wieder
  ausführen.
\end{itemize}

\textbf{Umgang mit Daten in R}

Das Bearbeiten und Ändern von Daten funktioniert mit R deutlich anders
als z.B. unter \emph{SPSS} -- schon weil es keine graphische Oberfläche
zum Bearbeiten der Daten gibt. Das ist aber \textbf{kein} Nachteil,
sondern ein Vorteil.

Der typische \emph{SPSS} Datensatz mit Variablen in den Spalten und
Fällen (Versuchspersonen) in den Zeilen heißt bei \textbf{R}
\textbf{Datenrahmen} (engl. \textbf{data frame}) und ist ein Objekt im
Arbeitsspeicher (engl. \emph{Workspace}). Ein Objekt kann entweder durch
das Einlesen einer Datendatei oder durch Zuweisung von Daten erzeugt
werden. Zum Einsatz kommt dabei der Zuweisungsbefehl
\texttt{\textless{}-}.

\hypertarget{datenmanagement-in-r}{%
\subsection{Datenmanagement in R}\label{datenmanagement-in-r}}

\textbf{Woher die Daten nehmen?}

Daten können auf unterschiedliche Art und Weise generiert werden:

\begin{itemize}
\tightlist
\item
  Erzeugen eigenen Daten
\item
  Einlesen von Sekundärdaten.
\item
  Einlesen von Daten aus Umfragestudie mittels Befragungssoftware oder
  händischer Eingabe.
\end{itemize}

\textbf{Eigene Daten erzeugen}

\begin{Shaded}
\begin{Highlighting}[]
\CommentTok{# Zwei Vektor erzeuchen}
\NormalTok{Bewertung<-}\KeywordTok{c}\NormalTok{(}\DecValTok{8}\NormalTok{,}\DecValTok{10}\NormalTok{,}\DecValTok{5}\NormalTok{,}\DecValTok{6}\NormalTok{,}\DecValTok{2}\NormalTok{) }\CommentTok{# nummeriche Daten}
\NormalTok{Geschlecht<-}\KeywordTok{c}\NormalTok{(}\StringTok{"m"}\NormalTok{,}\StringTok{"w"}\NormalTok{,}\StringTok{"m"}\NormalTok{,}\OtherTok{NA}\NormalTok{,}\StringTok{"w"}\NormalTok{) }\CommentTok{#kategoriale Daten}

\CommentTok{# Datenrahmen erzeugen}
\NormalTok{data<-}\KeywordTok{data.frame}\NormalTok{(Bewertung,Geschlecht)}

\CommentTok{# Gesamten Datenrahmen ausgeben}
\NormalTok{data}
\end{Highlighting}
\end{Shaded}

\begin{verbatim}
##   Bewertung Geschlecht
## 1         8          m
## 2        10          w
## 3         5          m
## 4         6       <NA>
## 5         2          w
\end{verbatim}

\begin{Shaded}
\begin{Highlighting}[]
\CommentTok{# Nur eine Spalte ausgeben}
\NormalTok{data[,}\DecValTok{2}\NormalTok{] }\CommentTok{# Spaltennummer reicht aus}
\end{Highlighting}
\end{Shaded}

\begin{verbatim}
## [1] m    w    m    <NA> w   
## Levels: m w
\end{verbatim}

\begin{Shaded}
\begin{Highlighting}[]
\CommentTok{# Eine bestimmte Zeile ausgeben}
\NormalTok{data[}\DecValTok{3}\NormalTok{,]}
\end{Highlighting}
\end{Shaded}

\begin{verbatim}
##   Bewertung Geschlecht
## 3         5          m
\end{verbatim}

\begin{Shaded}
\begin{Highlighting}[]
\CommentTok{# Inhalte einer Variablen fortlaufend ausgeben}
\NormalTok{data}\OperatorTok{$}\NormalTok{Geschlecht}
\end{Highlighting}
\end{Shaded}

\begin{verbatim}
## [1] m    w    m    <NA> w   
## Levels: m w
\end{verbatim}

\begin{Shaded}
\begin{Highlighting}[]
\CommentTok{# Variablen ändern (z. B. recodieren der 10er Skala)}
\NormalTok{data}\OperatorTok{$}\NormalTok{Bewertung <-}\StringTok{ }\DecValTok{11}\OperatorTok{-}\NormalTok{data}\OperatorTok{$}\NormalTok{Bewertung }\CommentTok{# (n+1)-X recodiert eine beliebige Skala}
\NormalTok{data[}\DecValTok{1}\NormalTok{]}
\end{Highlighting}
\end{Shaded}

\begin{verbatim}
##   Bewertung
## 1         3
## 2         1
## 3         6
## 4         5
## 5         9
\end{verbatim}

\begin{Shaded}
\begin{Highlighting}[]
\NormalTok{data}\OperatorTok{$}\NormalTok{Bewertung}
\end{Highlighting}
\end{Shaded}

\begin{verbatim}
## [1] 3 1 6 5 9
\end{verbatim}

\hypertarget{nutzliche-funktionen-in-r}{%
\subsection{Nützliche Funktionen in R}\label{nutzliche-funktionen-in-r}}

\begin{Shaded}
\begin{Highlighting}[]
\KeywordTok{length}\NormalTok{(data) }\CommentTok{# Anzahl der Elemente oder Komponenten}
\end{Highlighting}
\end{Shaded}

\begin{verbatim}
## [1] 2
\end{verbatim}

\begin{Shaded}
\begin{Highlighting}[]
\KeywordTok{str}\NormalTok{(data) }\CommentTok{# Struktur der Daten}
\end{Highlighting}
\end{Shaded}

\begin{verbatim}
## 'data.frame':    5 obs. of  2 variables:
##  $ Bewertung : num  3 1 6 5 9
##  $ Geschlecht: Factor w/ 2 levels "m","w": 1 2 1 NA 2
\end{verbatim}

\begin{Shaded}
\begin{Highlighting}[]
\KeywordTok{class}\NormalTok{(data) }\CommentTok{# Klasse oder Typ der Daten}
\end{Highlighting}
\end{Shaded}

\begin{verbatim}
## [1] "data.frame"
\end{verbatim}

\begin{Shaded}
\begin{Highlighting}[]
\KeywordTok{names}\NormalTok{(data) }\CommentTok{# Variablennamen}
\end{Highlighting}
\end{Shaded}

\begin{verbatim}
## [1] "Bewertung"  "Geschlecht"
\end{verbatim}

\begin{Shaded}
\begin{Highlighting}[]
\KeywordTok{c}\NormalTok{(}\DecValTok{1}\OperatorTok{:}\DecValTok{30}\NormalTok{) }\CommentTok{# kombinieren Daten in einen Vektor}
\end{Highlighting}
\end{Shaded}

\begin{verbatim}
##  [1]  1  2  3  4  5  6  7  8  9 10 11 12 13 14 15 16 17 18 19 20 21 22 23
## [24] 24 25 26 27 28 29 30
\end{verbatim}

\begin{Shaded}
\begin{Highlighting}[]
\KeywordTok{cbind}\NormalTok{(Bewertung, Geschlecht) }\CommentTok{# kombiniere die Daten als Spalten}
\end{Highlighting}
\end{Shaded}

\begin{verbatim}
##      Bewertung Geschlecht
## [1,] "8"       "m"       
## [2,] "10"      "w"       
## [3,] "5"       "m"       
## [4,] "6"       NA        
## [5,] "2"       "w"
\end{verbatim}

\begin{Shaded}
\begin{Highlighting}[]
\KeywordTok{rbind}\NormalTok{(Bewertung, Geschlecht) }\CommentTok{# kombiniere Daten als Zeilen}
\end{Highlighting}
\end{Shaded}

\begin{verbatim}
##            [,1] [,2] [,3] [,4] [,5]
## Bewertung  "8"  "10" "5"  "6"  "2" 
## Geschlecht "m"  "w"  "m"  NA   "w"
\end{verbatim}

\begin{Shaded}
\begin{Highlighting}[]
\KeywordTok{ls}\NormalTok{()     }\CommentTok{# Gibt alle aktiven Objekte aus}
\end{Highlighting}
\end{Shaded}

\begin{verbatim}
## [1] "Bewertung"  "data"       "Geschlecht"
\end{verbatim}

\begin{Shaded}
\begin{Highlighting}[]
\KeywordTok{rm}\NormalTok{(data) }\CommentTok{# Löscht Objekte}
\NormalTok{data<-}\KeywordTok{data.frame}\NormalTok{(Bewertung,Geschlecht) }\CommentTok{# und erzeugt sie wieder}
\end{Highlighting}
\end{Shaded}

\begin{Shaded}
\begin{Highlighting}[]
\NormalTok{newdata <-}\StringTok{ }\KeywordTok{edit}\NormalTok{(data)   }\CommentTok{# Editiert ein Objekt und speicht unter neuem Namen}
\KeywordTok{fix}\NormalTok{(data)               }\CommentTok{# edit in place}
\end{Highlighting}
\end{Shaded}

\hypertarget{arbeiten-mit-einem-neuen-datensatz}{%
\section{Arbeiten mit einem neuen
Datensatz}\label{arbeiten-mit-einem-neuen-datensatz}}

\hypertarget{daten-einlesen}{%
\subsection{Daten einlesen}\label{daten-einlesen}}

Es ist praktisch, wenn die eingelesenen Daten im gleichen Verzeichnis
liegen wie das Skript. Normalerweise arbeitet man nicht nur einmal mit
einem Datensatz, sondern immer wieder und lange Zeit und auf
unterschiedlichen Computern. Das Skript beinhaltet dabei alle
Informationen: * Welcher Datensatz wird eingelesen. * Was wird an den
Daten verändert. * Welche Verfahren werden mir dem eingelesenen
Datensatz duchgeführt. * et cetera, pp.

Wir verwenden für die Übungen den \texttt{tips} Datensatz. Dazu laden
wir die Daten auf unseren Rechner und importieren die Daten.

\begin{Shaded}
\begin{Highlighting}[]
\CommentTok{# Download der Daten mit dem Befehl}
\KeywordTok{download.file}\NormalTok{(}\StringTok{"https://goo.gl/whKjnl"}\NormalTok{, }\DataTypeTok{destfile =} \StringTok{"tips.csv"}\NormalTok{)}
\NormalTok{tips<-}\KeywordTok{read.csv2}\NormalTok{(}\StringTok{"tips.csv"}\NormalTok{)}
\end{Highlighting}
\end{Shaded}

Wenn die Daten nicht im gleichen Pfad wie das Arbeitsverzeichnis liegen,
dann muss das Arbeitsverzeichnis gesucht werden. Workspace suchen mit
getwd().

\begin{Shaded}
\begin{Highlighting}[]
\KeywordTok{getwd}\NormalTok{()}
\end{Highlighting}
\end{Shaded}

Wenn die Daten aus einem anderen Pfad als dem Worspace geladen werden
sollen, muss der Pfad angegeben werden:

\begin{Shaded}
\begin{Highlighting}[]
\KeywordTok{setwd}\NormalTok{(}\StringTok{"C:/.../..."}\NormalTok{)}
\end{Highlighting}
\end{Shaded}

\hypertarget{datenstruktur-prufen}{%
\subsection{Datenstruktur prüfen}\label{datenstruktur-prufen}}

Es empfiehlt sich, vor jeder Analyse den Datensatz zu betrachten. Von
besonderer Relevanz ist hier die Datenstruktur, d.h. Variablennamen und
Variablentypen, die Größe des Datensatzes sowie die ersten und letzten
paar Zeilen.

\begin{Shaded}
\begin{Highlighting}[]
\CommentTok{# Datenstruktur betrachten}
\KeywordTok{str}\NormalTok{(tips)}
\end{Highlighting}
\end{Shaded}

\begin{verbatim}
## 'data.frame':    244 obs. of  7 variables:
##  $ total_bill: num  17 10.3 21 23.7 24.6 ...
##  $ tip       : num  1.01 1.66 3.5 3.31 3.61 4.71 2 3.12 1.96 3.23 ...
##  $ sex       : Factor w/ 2 levels "Female","Male": 1 2 2 2 1 2 2 2 2 2 ...
##  $ smoker    : Factor w/ 2 levels "No","Yes": 1 1 1 1 1 1 1 1 1 1 ...
##  $ day       : Factor w/ 4 levels "Fri","Sat","Sun",..: 3 3 3 3 3 3 3 3 3 3 ...
##  $ time      : Factor w/ 2 levels "Dinner","Lunch": 1 1 1 1 1 1 1 1 1 1 ...
##  $ size      : int  2 3 3 2 4 4 2 4 2 2 ...
\end{verbatim}

\begin{Shaded}
\begin{Highlighting}[]
\CommentTok{# Dimensionen des Datensatzes}
\KeywordTok{dim}\NormalTok{(tips)}
\end{Highlighting}
\end{Shaded}

\begin{verbatim}
## [1] 244   7
\end{verbatim}

\begin{Shaded}
\begin{Highlighting}[]
\CommentTok{# Kopf oder Ende (tail) der Datenmatix betrachten}
\KeywordTok{head}\NormalTok{(tips)}
\end{Highlighting}
\end{Shaded}

\begin{verbatim}
##   total_bill  tip    sex smoker day   time size
## 1      16.99 1.01 Female     No Sun Dinner    2
## 2      10.34 1.66   Male     No Sun Dinner    3
## 3      21.01 3.50   Male     No Sun Dinner    3
## 4      23.68 3.31   Male     No Sun Dinner    2
## 5      24.59 3.61 Female     No Sun Dinner    4
## 6      25.29 4.71   Male     No Sun Dinner    4
\end{verbatim}

\begin{Shaded}
\begin{Highlighting}[]
\KeywordTok{tail}\NormalTok{(tips)}
\end{Highlighting}
\end{Shaded}

\begin{verbatim}
##     total_bill  tip    sex smoker  day   time size
## 239      35.83 4.67 Female     No  Sat Dinner    3
## 240      29.03 5.92   Male     No  Sat Dinner    3
## 241      27.18 2.00 Female    Yes  Sat Dinner    2
## 242      22.67 2.00   Male    Yes  Sat Dinner    2
## 243      17.82 1.75   Male     No  Sat Dinner    2
## 244      18.78 3.00 Female     No Thur Dinner    2
\end{verbatim}

\begin{Shaded}
\begin{Highlighting}[]
\CommentTok{# Ausgabe der levels bei kategorialen Variablen}
\KeywordTok{levels}\NormalTok{(tips}\OperatorTok{$}\NormalTok{sex)}
\end{Highlighting}
\end{Shaded}

\begin{verbatim}
## [1] "Female" "Male"
\end{verbatim}

\hypertarget{datenstruktur-verandern}{%
\subsection{Datenstruktur verändern}\label{datenstruktur-verandern}}

Normalerweise liegen neu erfasste Daten nicht so vor, wie wir sie für
die Datenanalyse benötigen. Wir müssen eventuell Daten konvertieren und
löschen. Beispielweise wird Geschlecht über ein Online-Befragungstool
meist nur mummerisch erfasst und die ZUweisung muss dann im Rahmen der
Datenbereinigung und -aufbereitung erfolgen. Wie wir Variablen
recodieren, haben wir bereits weiter oben gelernt.

\begin{Shaded}
\begin{Highlighting}[]
\CommentTok{# Das Geschlecht im Datensatz tips liegt bereits kategorial vor}
\NormalTok{tips}\OperatorTok{$}\NormalTok{sex}
\end{Highlighting}
\end{Shaded}

\begin{verbatim}
##   [1] Female Male   Male   Male   Female Male   Male   Male   Male   Male  
##  [11] Male   Female Male   Male   Female Male   Female Male   Female Male  
##  [21] Male   Female Female Male   Male   Male   Male   Male   Male   Female
##  [31] Male   Male   Female Female Male   Male   Male   Female Male   Male  
##  [41] Male   Male   Male   Male   Male   Male   Male   Male   Male   Male  
##  [51] Male   Female Female Male   Male   Male   Male   Female Male   Male  
##  [61] Male   Male   Male   Male   Male   Male   Female Female Male   Male  
##  [71] Male   Female Female Female Female Male   Male   Male   Male   Male  
##  [81] Male   Male   Female Male   Male   Female Male   Male   Male   Male  
##  [91] Male   Male   Female Female Female Male   Male   Male   Male   Male  
## [101] Female Female Female Female Female Male   Male   Male   Male   Female
## [111] Male   Female Male   Male   Female Female Male   Female Female Female
## [121] Male   Female Male   Male   Female Female Male   Female Female Male  
## [131] Male   Female Female Female Female Female Female Female Male   Female
## [141] Female Male   Male   Female Female Female Female Female Male   Male  
## [151] Male   Male   Male   Male   Male   Female Male   Female Female Male  
## [161] Male   Male   Female Male   Female Male   Male   Male   Female Female
## [171] Male   Male   Male   Male   Male   Male   Male   Male   Female Male  
## [181] Male   Male   Male   Male   Male   Male   Female Male   Female Male  
## [191] Male   Female Male   Male   Male   Male   Male   Female Female Male  
## [201] Male   Female Female Female Male   Female Male   Male   Male   Female
## [211] Male   Male   Male   Female Female Female Male   Male   Male   Female
## [221] Male   Female Male   Female Male   Female Female Male   Male   Female
## [231] Male   Male   Male   Male   Male   Male   Male   Male   Female Male  
## [241] Female Male   Male   Female
## Levels: Female Male
\end{verbatim}

\begin{Shaded}
\begin{Highlighting}[]
\CommentTok{# Ausgabe der levels}
\KeywordTok{levels}\NormalTok{(tips}\OperatorTok{$}\NormalTok{sex)}
\end{Highlighting}
\end{Shaded}

\begin{verbatim}
## [1] "Female" "Male"
\end{verbatim}

\begin{Shaded}
\begin{Highlighting}[]
\CommentTok{# Konvertieren von factor auf nummerisch}
\NormalTok{tips}\OperatorTok{$}\NormalTok{sex<-}\KeywordTok{as.numeric}\NormalTok{(tips}\OperatorTok{$}\NormalTok{sex)}
\NormalTok{tips}\OperatorTok{$}\NormalTok{sex}
\end{Highlighting}
\end{Shaded}

\begin{verbatim}
##   [1] 1 2 2 2 1 2 2 2 2 2 2 1 2 2 1 2 1 2 1 2 2 1 1 2 2 2 2 2 2 1 2 2 1 1 2
##  [36] 2 2 1 2 2 2 2 2 2 2 2 2 2 2 2 2 1 1 2 2 2 2 1 2 2 2 2 2 2 2 2 1 1 2 2
##  [71] 2 1 1 1 1 2 2 2 2 2 2 2 1 2 2 1 2 2 2 2 2 2 1 1 1 2 2 2 2 2 1 1 1 1 1
## [106] 2 2 2 2 1 2 1 2 2 1 1 2 1 1 1 2 1 2 2 1 1 2 1 1 2 2 1 1 1 1 1 1 1 2 1
## [141] 1 2 2 1 1 1 1 1 2 2 2 2 2 2 2 1 2 1 1 2 2 2 1 2 1 2 2 2 1 1 2 2 2 2 2
## [176] 2 2 2 1 2 2 2 2 2 2 2 1 2 1 2 2 1 2 2 2 2 2 1 1 2 2 1 1 1 2 1 2 2 2 1
## [211] 2 2 2 1 1 1 2 2 2 1 2 1 2 1 2 1 1 2 2 1 2 2 2 2 2 2 2 2 1 2 1 2 2 1
\end{verbatim}

\begin{Shaded}
\begin{Highlighting}[]
\KeywordTok{levels}\NormalTok{(tips}\OperatorTok{$}\NormalTok{sex) }\CommentTok{# Es gibt keine, da jetzt nummerisch!}
\end{Highlighting}
\end{Shaded}

\begin{verbatim}
## NULL
\end{verbatim}

\begin{Shaded}
\begin{Highlighting}[]
\CommentTok{# Konvertieren von nummerisch auf factor}
\NormalTok{tips}\OperatorTok{$}\NormalTok{sex<-}\KeywordTok{as.factor}\NormalTok{(tips}\OperatorTok{$}\NormalTok{sex)}
\KeywordTok{levels}\NormalTok{(tips}\OperatorTok{$}\NormalTok{sex)}
\end{Highlighting}
\end{Shaded}

\begin{verbatim}
## [1] "1" "2"
\end{verbatim}

\begin{Shaded}
\begin{Highlighting}[]
\CommentTok{# Wertzuweisung von levels}
\KeywordTok{levels}\NormalTok{(tips}\OperatorTok{$}\NormalTok{sex)<-}\KeywordTok{c}\NormalTok{(}\StringTok{"f"}\NormalTok{,}\StringTok{"m"}\NormalTok{)}
\KeywordTok{levels}\NormalTok{(tips}\OperatorTok{$}\NormalTok{sex)}
\end{Highlighting}
\end{Shaded}

\begin{verbatim}
## [1] "f" "m"
\end{verbatim}

\begin{Shaded}
\begin{Highlighting}[]
\NormalTok{tips}\OperatorTok{$}\NormalTok{sex}
\end{Highlighting}
\end{Shaded}

\begin{verbatim}
##   [1] f m m m f m m m m m m f m m f m f m f m m f f m m m m m m f m m f f m
##  [36] m m f m m m m m m m m m m m m m f f m m m m f m m m m m m m m f f m m
##  [71] m f f f f m m m m m m m f m m f m m m m m m f f f m m m m m f f f f f
## [106] m m m m f m f m m f f m f f f m f m m f f m f f m m f f f f f f f m f
## [141] f m m f f f f f m m m m m m m f m f f m m m f m f m m m f f m m m m m
## [176] m m m f m m m m m m m f m f m m f m m m m m f f m m f f f m f m m m f
## [211] m m m f f f m m m f m f m f m f f m m f m m m m m m m m f m f m m f
## Levels: f m
\end{verbatim}

\begin{Shaded}
\begin{Highlighting}[]
\CommentTok{# oder alternativ alles in einem Schritt!}
\CommentTok{# Dabei wird die Spalte sex wieder überschrieben}
\NormalTok{tips}\OperatorTok{$}\NormalTok{sex<-}\KeywordTok{factor}\NormalTok{(tips}\OperatorTok{$}\NormalTok{sex, }\DataTypeTok{labels =} \KeywordTok{c}\NormalTok{(}\StringTok{"female"}\NormalTok{, }\StringTok{"male"}\NormalTok{)) }
\KeywordTok{levels}\NormalTok{(tips}\OperatorTok{$}\NormalTok{sex)}
\end{Highlighting}
\end{Shaded}

\begin{verbatim}
## [1] "female" "male"
\end{verbatim}

\hypertarget{arbeiten-mit-dem-paket-mosaic}{%
\section{\texorpdfstring{Arbeiten mit dem Paket
\texttt{mosaic}}{Arbeiten mit dem Paket mosaic}}\label{arbeiten-mit-dem-paket-mosaic}}

Die folgende Syntax

\begin{quote}
Zielbefehl(y \textasciitilde{} x \textbar{} z, data=\ldots{})
\end{quote}

wird verwendet für

\begin{itemize}
\tightlist
\item
  graphische Zusammenfassungen,
\item
  numerische Zusammenfassungen und
\item
  inferentstatistische Auswertungen
\end{itemize}

Für Grafiken gilt:

\begin{itemize}
\tightlist
\item
  y: y-Achse Variable
\item
  x: x-Achse Variable
\item
  z: Bedingungsvariable
\end{itemize}

Generell gilt:

\begin{quote}
\(y ~ x | z\) analog zu \(y = f(x)|_z\)
\end{quote}

kann in der Regel gelesen werden \textbf{y wird modelliert von (oder
hängt ab von) x unterschieden für jedes z}.

Laden der \texttt{mosaic} Paktes

\begin{Shaded}
\begin{Highlighting}[]
\KeywordTok{library}\NormalTok{(mosaic)}
\end{Highlighting}
\end{Shaded}

\begin{verbatim}
## Warning: package 'mosaic' was built under R version 3.4.1
\end{verbatim}

\begin{verbatim}
## Loading required package: dplyr
\end{verbatim}

\begin{verbatim}
## Warning: package 'dplyr' was built under R version 3.4.1
\end{verbatim}

\begin{verbatim}
## 
## Attaching package: 'dplyr'
\end{verbatim}

\begin{verbatim}
## The following objects are masked from 'package:stats':
## 
##     filter, lag
\end{verbatim}

\begin{verbatim}
## The following objects are masked from 'package:base':
## 
##     intersect, setdiff, setequal, union
\end{verbatim}

\begin{verbatim}
## Loading required package: lattice
\end{verbatim}

\begin{verbatim}
## Loading required package: ggformula
\end{verbatim}

\begin{verbatim}
## Warning: package 'ggformula' was built under R version 3.4.1
\end{verbatim}

\begin{verbatim}
## Loading required package: ggplot2
\end{verbatim}

\begin{verbatim}
## 
## New to ggformula?  Try the tutorials: 
##  learnr::run_tutorial("introduction", package = "ggformula")
##  learnr::run_tutorial("refining", package = "ggformula")
\end{verbatim}

\begin{verbatim}
## Loading required package: mosaicData
\end{verbatim}

\begin{verbatim}
## Loading required package: Matrix
\end{verbatim}

\begin{verbatim}
## Warning: package 'Matrix' was built under R version 3.4.1
\end{verbatim}

\begin{verbatim}
## 
## The 'mosaic' package masks several functions from core packages in order to add 
## additional features.  The original behavior of these functions should not be affected by this.
## 
## Note: If you use the Matrix package, be sure to load it BEFORE loading mosaic.
\end{verbatim}

\begin{verbatim}
## 
## Attaching package: 'mosaic'
\end{verbatim}

\begin{verbatim}
## The following object is masked from 'package:Matrix':
## 
##     mean
\end{verbatim}

\begin{verbatim}
## The following objects are masked from 'package:dplyr':
## 
##     count, do, tally
\end{verbatim}

\begin{verbatim}
## The following objects are masked from 'package:stats':
## 
##     binom.test, cor, cor.test, cov, fivenum, IQR, median,
##     prop.test, quantile, sd, t.test, var
\end{verbatim}

\begin{verbatim}
## The following objects are masked from 'package:base':
## 
##     max, mean, min, prod, range, sample, sum
\end{verbatim}

\hypertarget{grafische-verfahren}{%
\section{Grafische Verfahren}\label{grafische-verfahren}}

\hypertarget{balkendiagramme-und-boxplots}{%
\subsection{Balkendiagramme und
Boxplots}\label{balkendiagramme-und-boxplots}}

\begin{Shaded}
\begin{Highlighting}[]
\CommentTok{#Balkendiagramm bei kategorialen Daten}
\KeywordTok{bargraph}\NormalTok{(}\OperatorTok{~}\NormalTok{sex, }\DataTypeTok{data=}\NormalTok{tips)}
\end{Highlighting}
\end{Shaded}

\includegraphics{Datenhandling_und_Einstieg_in_die_Analyse_mit_R_files/figure-latex/unnamed-chunk-10-1.pdf}

\begin{Shaded}
\begin{Highlighting}[]
\CommentTok{#Histogramm bei metirschen Daten}
\KeywordTok{histogram}\NormalTok{(}\OperatorTok{~}\NormalTok{tip, }\DataTypeTok{data=}\NormalTok{tips)}
\end{Highlighting}
\end{Shaded}

\includegraphics{Datenhandling_und_Einstieg_in_die_Analyse_mit_R_files/figure-latex/unnamed-chunk-10-2.pdf}

\begin{Shaded}
\begin{Highlighting}[]
\CommentTok{#mit Geschlechtertrennung}
\KeywordTok{histogram}\NormalTok{(}\OperatorTok{~}\NormalTok{tip }\OperatorTok{|}\StringTok{ }\NormalTok{sex, }\DataTypeTok{data=}\NormalTok{tips)}
\end{Highlighting}
\end{Shaded}

\includegraphics{Datenhandling_und_Einstieg_in_die_Analyse_mit_R_files/figure-latex/unnamed-chunk-10-3.pdf}

\begin{Shaded}
\begin{Highlighting}[]
\CommentTok{#Boxplot mit metrischen Daten}
\KeywordTok{bwplot}\NormalTok{(}\OperatorTok{~}\NormalTok{tip, }\DataTypeTok{data=}\NormalTok{tips)}
\end{Highlighting}
\end{Shaded}

\includegraphics{Datenhandling_und_Einstieg_in_die_Analyse_mit_R_files/figure-latex/unnamed-chunk-10-4.pdf}

\begin{Shaded}
\begin{Highlighting}[]
\CommentTok{#Boxplot mit metrischen Daten für Gruppem}
\KeywordTok{bwplot}\NormalTok{(}\OperatorTok{~}\NormalTok{tip }\OperatorTok{|}\NormalTok{sex, }\DataTypeTok{data=}\NormalTok{tips)}
\end{Highlighting}
\end{Shaded}

\includegraphics{Datenhandling_und_Einstieg_in_die_Analyse_mit_R_files/figure-latex/unnamed-chunk-10-5.pdf}

\begin{Shaded}
\begin{Highlighting}[]
\CommentTok{#oder}
\KeywordTok{bwplot}\NormalTok{(tip}\OperatorTok{~}\NormalTok{sex , }\DataTypeTok{data=}\NormalTok{tips)}
\end{Highlighting}
\end{Shaded}

\includegraphics{Datenhandling_und_Einstieg_in_die_Analyse_mit_R_files/figure-latex/unnamed-chunk-10-6.pdf}

\begin{Shaded}
\begin{Highlighting}[]
\KeywordTok{bwplot}\NormalTok{(tip}\OperatorTok{~}\NormalTok{sex }\OperatorTok{|}\NormalTok{smoker, }\DataTypeTok{data=}\NormalTok{tips)}
\end{Highlighting}
\end{Shaded}

\includegraphics{Datenhandling_und_Einstieg_in_die_Analyse_mit_R_files/figure-latex/unnamed-chunk-10-7.pdf}

\begin{Shaded}
\begin{Highlighting}[]
\CommentTok{#Scatterplot (Streudiagramm) mit zwei metrischen Variablen}
\KeywordTok{xyplot}\NormalTok{(tip}\OperatorTok{~}\NormalTok{total_bill, }\DataTypeTok{data=}\NormalTok{tips)}
\end{Highlighting}
\end{Shaded}

\includegraphics{Datenhandling_und_Einstieg_in_die_Analyse_mit_R_files/figure-latex/unnamed-chunk-10-8.pdf}

\begin{Shaded}
\begin{Highlighting}[]
\CommentTok{#Plot der Regressiongeraden}
\KeywordTok{plotModel}\NormalTok{(}\KeywordTok{lm}\NormalTok{(tip}\OperatorTok{~}\NormalTok{total_bill, }\DataTypeTok{data=}\NormalTok{tips))}
\end{Highlighting}
\end{Shaded}

\includegraphics{Datenhandling_und_Einstieg_in_die_Analyse_mit_R_files/figure-latex/unnamed-chunk-10-9.pdf}

\hypertarget{tabellen}{%
\subsection{Tabellen}\label{tabellen}}

\begin{Shaded}
\begin{Highlighting}[]
\CommentTok{#mosaicplot mit zwei kategorialen Variablen}
\CommentTok{#Vorher muss eine Tabelle mit dem Befehl tally generiert werden}
\KeywordTok{tally}\NormalTok{(sex}\OperatorTok{~}\NormalTok{smoker, }\DataTypeTok{data=}\NormalTok{tips)}
\end{Highlighting}
\end{Shaded}

\begin{verbatim}
##         smoker
## sex      No Yes
##   female 54  33
##   male   97  60
\end{verbatim}

\begin{Shaded}
\begin{Highlighting}[]
\CommentTok{#Tablele wir einem Objet zugewiesen}
\NormalTok{Tabelle1 <-}\StringTok{ }\KeywordTok{tally}\NormalTok{(sex}\OperatorTok{~}\NormalTok{smoker, }\DataTypeTok{data=}\NormalTok{tips)}

\CommentTok{#Mit Tabelle1 kann nun ein mosaci plot generiert werden}
\KeywordTok{mosaicplot}\NormalTok{(Tabelle1)}
\end{Highlighting}
\end{Shaded}

\includegraphics{Datenhandling_und_Einstieg_in_die_Analyse_mit_R_files/figure-latex/unnamed-chunk-11-1.pdf}

\begin{Shaded}
\begin{Highlighting}[]
\CommentTok{#...oder alles zusammen (Das gilt übrigend grundsätzlich: Es kann über }
\CommentTok{#die Objektebene gegangen werden oder der Befehle direkt ausgeführt werden}
\KeywordTok{mosaicplot}\NormalTok{(}\KeywordTok{tally}\NormalTok{(sex}\OperatorTok{~}\NormalTok{smoker, }\DataTypeTok{data=}\NormalTok{tips))}
\end{Highlighting}
\end{Shaded}

\includegraphics{Datenhandling_und_Einstieg_in_die_Analyse_mit_R_files/figure-latex/unnamed-chunk-11-2.pdf}

\hypertarget{kennzahlen}{%
\section{Kennzahlen}\label{kennzahlen}}

\begin{Shaded}
\begin{Highlighting}[]
\CommentTok{# Mittelwert}
\KeywordTok{mean}\NormalTok{(tip}\OperatorTok{~}\NormalTok{sex, }\DataTypeTok{data=}\NormalTok{tips)}
\end{Highlighting}
\end{Shaded}

\begin{verbatim}
##   female     male 
## 2.833448 3.089618
\end{verbatim}

\begin{Shaded}
\begin{Highlighting}[]
\CommentTok{#Anstatt mean können alle Lageparameter oder Streumaße erechnet werden }
\KeywordTok{median}\NormalTok{(tip}\OperatorTok{~}\NormalTok{sex, }\DataTypeTok{data=}\NormalTok{tips)}
\end{Highlighting}
\end{Shaded}

\begin{verbatim}
## female   male 
##   2.75   3.00
\end{verbatim}

\begin{Shaded}
\begin{Highlighting}[]
\KeywordTok{sd}\NormalTok{(tip}\OperatorTok{~}\NormalTok{sex, }\DataTypeTok{data=}\NormalTok{tips)}
\end{Highlighting}
\end{Shaded}

\begin{verbatim}
##   female     male 
## 1.159495 1.489102
\end{verbatim}

\begin{Shaded}
\begin{Highlighting}[]
\KeywordTok{var}\NormalTok{(tip}\OperatorTok{~}\NormalTok{sex, }\DataTypeTok{data=}\NormalTok{tips)}
\end{Highlighting}
\end{Shaded}

\begin{verbatim}
##   female     male 
## 1.344428 2.217424
\end{verbatim}

\begin{Shaded}
\begin{Highlighting}[]
\KeywordTok{IQR}\NormalTok{(tip}\OperatorTok{~}\NormalTok{sex, }\DataTypeTok{data=}\NormalTok{tips)}
\end{Highlighting}
\end{Shaded}

\begin{verbatim}
## female   male 
##   1.50   1.76
\end{verbatim}

\begin{Shaded}
\begin{Highlighting}[]
\KeywordTok{diffmean}\NormalTok{(tip}\OperatorTok{~}\NormalTok{sex, }\DataTypeTok{data=}\NormalTok{tips)}
\end{Highlighting}
\end{Shaded}

\begin{verbatim}
##  diffmean 
## 0.2561696
\end{verbatim}

\begin{Shaded}
\begin{Highlighting}[]
\KeywordTok{min}\NormalTok{(tip}\OperatorTok{~}\NormalTok{sex, }\DataTypeTok{data=}\NormalTok{tips)}
\end{Highlighting}
\end{Shaded}

\begin{verbatim}
## female   male 
##      1      1
\end{verbatim}

\begin{Shaded}
\begin{Highlighting}[]
\KeywordTok{max}\NormalTok{(tip}\OperatorTok{~}\NormalTok{sex, }\DataTypeTok{data=}\NormalTok{tips)}
\end{Highlighting}
\end{Shaded}

\begin{verbatim}
## female   male 
##    6.5   10.0
\end{verbatim}

\begin{Shaded}
\begin{Highlighting}[]
\CommentTok{#Favorisierte Statistiken werden ausgegeben mit}
\KeywordTok{favstats}\NormalTok{(tip}\OperatorTok{~}\NormalTok{sex, }\DataTypeTok{data=}\NormalTok{tips)}
\end{Highlighting}
\end{Shaded}

\begin{verbatim}
##      sex min Q1 median   Q3  max     mean       sd   n missing
## 1 female   1  2   2.75 3.50  6.5 2.833448 1.159495  87       0
## 2   male   1  2   3.00 3.76 10.0 3.089618 1.489102 157       0
\end{verbatim}

\begin{Shaded}
\begin{Highlighting}[]
\CommentTok{#Korrelation als Zusammenhangsmaß mit metrischen Variablen}
\KeywordTok{cor}\NormalTok{(tip}\OperatorTok{~}\NormalTok{total_bill, }\DataTypeTok{data=}\NormalTok{tips)}
\end{Highlighting}
\end{Shaded}

\begin{verbatim}
## [1] 0.6757341
\end{verbatim}

\begin{Shaded}
\begin{Highlighting}[]
\KeywordTok{cor}\NormalTok{(tips[,}\KeywordTok{c}\NormalTok{(}\StringTok{"total_bill"}\NormalTok{, }\StringTok{"tip"}\NormalTok{, }\StringTok{"size"}\NormalTok{)])}
\end{Highlighting}
\end{Shaded}

\begin{verbatim}
##            total_bill       tip      size
## total_bill  1.0000000 0.6757341 0.5983151
## tip         0.6757341 1.0000000 0.4892988
## size        0.5983151 0.4892988 1.0000000
\end{verbatim}

\begin{center}\rule{0.5\linewidth}{\linethickness}\end{center}

\textbf{Tip}

\begin{quote}
Wenn über mehrere Variablen eine Dimension (durch z. B. mean) gebildet
werden soll, dann eignet sich für die Aggregation der Daten der apply
Befehl.(Achtung: hier mit tips nicht möglich!)
\end{quote}

\begin{quote}
Datensatz\$Neue\_Variable \textless{}-
apply(Datensatz{[},c(``Variable1'',``Variable2'',
``etc..''){]},1,mean,na.rm=TRUE)
\end{quote}

\begin{center}\rule{0.5\linewidth}{\linethickness}\end{center}

\hypertarget{korrleationsplot}{%
\section{Korrleationsplot}\label{korrleationsplot}}

Mit Hilfe des Zusatzpakets \texttt{corrplot} lassen sich Korrelationen
besonders einfach visualisieren.

\begin{Shaded}
\begin{Highlighting}[]
\CommentTok{#Zusatzpaket laden}
\KeywordTok{library}\NormalTok{(corrplot)}

\KeywordTok{corrplot}\NormalTok{(}\KeywordTok{cor}\NormalTok{(tips[,}\KeywordTok{c}\NormalTok{(}\StringTok{"total_bill"}\NormalTok{, }\StringTok{"tip"}\NormalTok{, }\StringTok{"size"}\NormalTok{)]))}
\end{Highlighting}
\end{Shaded}

\includegraphics{Datenhandling_und_Einstieg_in_die_Analyse_mit_R_files/figure-latex/unnamed-chunk-13-1.pdf}

Je intensiver die Farbe, desto höher die Korrelation. Hier gibt es
unzählige Einstellmöglichkeiten, siehe \texttt{?corrplot} bzw. für
Beispiele:

\begin{Shaded}
\begin{Highlighting}[]
\KeywordTok{vignette}\NormalTok{(}\StringTok{"corrplot-intro"}\NormalTok{)}
\end{Highlighting}
\end{Shaded}

Noch einfacher, aber nicht so schön geht es mit dem Paket
\texttt{corrgram}. Hier müssen nicht extra die metrischen Variablen
ausgewählt werden. Er nimmt nur alle metrischen Variablen im Datensatz
mit.

\begin{Shaded}
\begin{Highlighting}[]
\KeywordTok{library}\NormalTok{(corrgram)}
\KeywordTok{corrgram}\NormalTok{(tips)}
\end{Highlighting}
\end{Shaded}

\includegraphics{Datenhandling_und_Einstieg_in_die_Analyse_mit_R_files/figure-latex/unnamed-chunk-15-1.pdf}

\newpage

\hypertarget{literaturempfehlung-fur-den-einstieg-in-r-mit-dem-paket-mosaic}{%
\section{Literaturempfehlung für den Einstieg in R mit dem Paket
mosaic}\label{literaturempfehlung-fur-den-einstieg-in-r-mit-dem-paket-mosaic}}

\begin{itemize}
\tightlist
\item
  Daniel T. Kaplan, Nicholas J. Horton, Randall Pruim, (2013): Project
  MOSAIC Little Books \emph{Start Teaching with R},
  \url{http://mosaic-web.org/go/Master-Starting.pdf}
\end{itemize}

\hypertarget{versionshinweise}{%
\section{Versionshinweise:}\label{versionshinweise}}

\begin{itemize}
\tightlist
\item
  Autor des Orginal-Skripts: Prof.~Dr.~Oliver Gansser

  \begin{itemize}
  \tightlist
  \item
    Danke für das Überlassen!
  \end{itemize}
\item
  Datum erstellt: 2017-09-23
\item
  R Version: 3.4.0
\item
  \texttt{mosaic} Version: 1.1.0
\end{itemize}


\end{document}
